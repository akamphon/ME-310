% \usepackage[no-math]{fontspec} % set input encoding (not needed with XeLaTeX)
\setcounter{secnumdepth}{2}
\setcounter{tocdepth}{1}


%%% Examples of Article customizations

%%% PAGE DIMENSIONS
\usepackage{geometry} % to change the page dimensions
\geometry{a4paper} % or letterpaper (US) or a5paper or....
\geometry{margin=1in} % for example, change the margins to 2 inches all round
% \geometry{landscape} % set up the page for landscape

\usepackage{graphicx} % support the \includegraphics command and options

\usepackage[parfill]{parskip} % Activate to begin paragraphs with an empty line rather than an indent

%%% PACKAGES
\usepackage{booktabs} % for much better looking tables
\usepackage{array} % for better arrays (eg matrices) in maths
\usepackage{enumitem}
\usepackage{subfig} % make it possible to include more than one captioned figure/table in a single float
\usepackage{colortbl}

%%% HEADERS & FOOTERS
\usepackage{fancyhdr} % This should be set AFTER setting up the page geometry
\pagestyle{fancy} % options: empty , plain , fancy
\renewcommand{\headrulewidth}{1pt} % customise the layout...
\renewcommand{\footrulewidth}{1pt}
\lhead{\rightmark}\chead{}\rhead{}
\lfoot{ME 310: Mechanical Design I}\cfoot{\thepage}\rfoot{S. Akamphon}

%%% ToC (table of contents) APPEARANCE
\usepackage[numbib]{tocbibind} % Put the bibliography in the ToC
\usepackage[titles,subfigure]{tocloft} % Alter the style of the Table of Contents
\renewcommand{\cftchapfont}{\bfseries\rmfamily}
\newcommand*\chem[1]{\textup{#1}}

\usepackage{hyperref}
\hypersetup{
  colorlinks,
  citecolor=black,
  filecolor=black,
  linkcolor=black,
  urlcolor=black
}

\let\openbox\relax
\usepackage{amsthm}
\usepackage{bm}
\usepackage{multirow}
\usepackage{gensymb}
\usepackage[nointegrals]{wasysym}
\usepackage{array}
\usepackage{siunitx}
\usepackage{newtxtext}
\usepackage{newtxmath}
\usepackage{afterpage}

\newcommand\blankpage{%
    \null
    \thispagestyle{empty}%
    \addtocounter{page}{-1}%
    \newpage}

%% Chapter Heading %%%%%
\usepackage{titlesec}

%% Drawing pics with tikz %%%
\usepackage{tikz}
\usetikzlibrary{arrows,calc,decorations,shapes,decorations.pathmorphing,patterns}
\tikzset{>=latex}
\usepackage{pgfplots}
\newcommand{\AxisRotator}[1][rotate=0]{%
    \tikz[x=0.25cm,y=0.60cm,line width=.2ex,-stealth,#1] \draw(0,0) arc (150:-150:1 and 1);%
}

\definecolor{lightblue}{RGB}{180,220,255}
\definecolor{titlepagecolor}{cmyk}{1,.60,0,.40}
\definecolor{namecolor}{cmyk}{1,.50,0,.10}

\titleformat{\chapter}
{\huge\bfseries}
{\thechapter}{1em}{}

%% Section Heading %%%%%

\titleformat{\section}
{\Large\bfseries}
{\thesection}{1em}{}
   
%% Subsection Heading %%%%

\titleformat{\subsection}
{\large\bfseries}
{\thesubsection}{1em}{}

\titleformat{\subsubsection}
{\bfseries}
{\thesubsubsection}{1em}{}

%%% New column type
\newcolumntype{L}[1]{>{\raggedright\let\newline\\\arraybackslash\hspace{0pt}}m{#1}}
\newcolumntype{C}[1]{>{\centering\let\newline\\\arraybackslash\hspace{0pt}}m{#1}}
\newcolumntype{R}[1]{>{\raggedleft\let\newline\\\arraybackslash\hspace{0pt}}m{#1}}

%%% Example environment
\usepackage{thmtools}
\usepackage{mdframed}
\usepackage{float}
\declaretheoremstyle[
spaceabove=6pt, spacebelow=6pt,
headfont=\normalfont\bfseries,
notefont=\mdseries, notebraces={(}{)},
bodyfont=\normalfont,
postheadspace=1em,
numberwithin=chapter,
preheadhook={\begin{mdframed}[backgroundcolor=lightblue,
    innertopmargin=6pt , splittopskip=\topskip, % 
    skipbelow= 0pt, skipabove=6pt, %
    topline=false,bottomline=false,leftline=false,rightline=false] \sloppy},
  postfoothook=\end{mdframed},
headpunct={}
]{exstyle}
\declaretheoremstyle[
spaceabove=6pt, spacebelow=6pt,
headfont=\normalfont\bfseries,
notefont=\mdseries, notebraces={(}{)},
bodyfont=\normalfont,
postheadspace=1em,
preheadhook={\begin{mdframed}[backgroundcolor=lightblue,
    innertopmargin =6pt , splittopskip = \topskip, % 
    skipbelow=6pt, skipabove=0pt, %
    topline=false,bottomline=false,leftline=false,rightline=false] \sloppy},
  postfoothook=\end{mdframed},
headpunct={},
numbered=no
]{solstyle}
\declaretheorem[style=exstyle]{example}
\declaretheorem[style=solstyle]{solution}

% New list for exercises
\newlist{exercises}{enumerate}{2}
\setlist[exercises]{%
  label=\textbf{\thechapter-\arabic*}~,  % Label: Exercise C.E
  ref=\thechapter-\arabic*, % References: C.E (important!)
  align=left,               % Left align labels
  labelindent=0pt,          % No space betw. margin of list and label
  leftmargin=0pt,           % No space betw. margin of list and following lines
  itemindent=!,             % Indention of item computet automatically
}

%\setlist[enumerate, 1]{label=(\alph*)}            % Label for subexercises

\newcommand{\exercise}{%
\item \label{lab:\arabic{chapter}.\arabic{exercisesi}}  % Append label to item
}

\SetLabelAlign{parright}{\parbox[t]{\labelwidth}{\raggedleft#1}}

\setlist[description]{style=multiline,leftmargin=5cm,font=\bfseries,%
    align=parright}

\usepackage{cleveref}

% New decoration for screws
\tikzset{/pgf/decoration/.cd,
    head width/.initial=6pt,
    head length/.initial=1.5pt,
    thread separation/.initial=1.0pt,
    thread amplitude/.initial=0.5pt,
    screw radius/.initial=1.2pt,
}
% definition of the decoration
\pgfdeclaredecoration{screw}{initial}
{
  \state{initial}[width=\pgfkeysvalueof{/pgf/decoration/head length},%
  next state=midd]
  {
    \def\headlength{%
      \pgfkeysvalueof{/pgf/decoration/head length}%
    }
    \def\headwidth{%
      \pgfkeysvalueof{/pgf/decoration/head width}%
    }
    \def\screwradius{%
      \pgfkeysvalueof{/pgf/decoration/screw radius}%
    }
    % First line
    \pgfpathlineto{\pgfpoint{0.0pt}{\headwidth/2}}
    \pgfpathlineto{\pgfpoint{\headlength}{\headwidth/2}}
    \pgfpathlineto{\pgfpoint{\headlength}{\screwradius}}
    % Second line
    \pgfpathmoveto{\pgfpoint{0.0pt}{0.0pt}}
    \pgfpathlineto{\pgfpoint{0.0pt}{-\headwidth/2}}
    \pgfpathlineto{\pgfpoint{\headlength}{-\headwidth/2}}
    \pgfpathlineto{\pgfpoint{\headlength}{-\screwradius}}
  }
  \state{midd}[width=\pgfkeysvalueof{/pgf/decoration/thread separation}*2]
  {
    \def\threadseparation{%
      \pgfkeysvalueof{/pgf/decoration/thread separation}%
    }
    \def\threadamplitude{%
      \pgfkeysvalueof{/pgf/decoration/thread amplitude}%
    }
    \def\screwradius{%
      \pgfkeysvalueof{/pgf/decoration/screw radius}%
    }
    % First line
    \pgfpathmoveto{\pgfpoint{0pt}{\screwradius}}
    \pgfpathlineto{\pgfpoint{0.5*\threadseparation}{\screwradius+\threadamplitude}}
    \pgfpathlineto{\pgfpoint{1.0*\threadseparation}{\screwradius}}
    \pgfpathlineto{\pgfpoint{1.5*\threadseparation}{\screwradius-\threadamplitude}}
    \pgfpathlineto{\pgfpoint{2.0*\threadseparation}{\screwradius}}
    % Second line
    \pgfpathmoveto{\pgfpoint{0pt}{-\screwradius}}
    \pgfpathlineto{\pgfpoint{0.5*\threadseparation}{-\screwradius-\threadamplitude}}
    \pgfpathlineto{\pgfpoint{1.0*\threadseparation}{-\screwradius}}
    \pgfpathlineto{\pgfpoint{1.5*\threadseparation}{-\screwradius+\threadamplitude}}
    \pgfpathlineto{\pgfpoint{2.0*\threadseparation}{-\screwradius}}
    % Thread
    \pgfpathmoveto{\pgfpoint{0.5*\threadseparation}{\screwradius+\threadamplitude}}
    \pgfpathlineto{\pgfpoint{1.5*\threadseparation}{-\screwradius+\threadamplitude}}
  }
  \state{final}
  {
    \def\screwradius{%
      \pgfkeysvalueof{/pgf/decoration/screw radius}%
    }
    %\pgfpathlineto{\pgfpointdecoratedpathlast}
    \pgfpathmoveto{\pgfpoint{0pt}{\screwradius}}
    \pgfpathlineto{\pgfpoint{\screwradius/2}{0pt}}
    \pgfpathlineto{\pgfpoint{0pt}{-\screwradius}}
  }
}

%% bibliography %%%%

\usepackage[style=numeric,backend=biber]{biblatex}
\addbibresource{me310.bib}

%%% END Article customizations